%%%%%%%%%%%%%%%%%%%%%%%%%%%%
% ABSTRACT
%%%%%%%%%%%%%%%%%%%%%%%%%%%%

\chapter*{Abstract}

With the growth in Atrato’s customers and partners consuming its main financial product, the number of loans granted on a daily basis has increased and the manual money disbursement process has become an exhaustive, repetitive, and error-prone task. In order to ensure a scalable and reliable disbursement procedure, this should progressively migrate to a fully automated activity compatible with current processes. To track and monitor the complete disbursement procedure, a Balance System was built enabling an understanding of both a merchant's balance in a general and in a very granular way, implementing a logging system to ensure detailed visibility and traceability of the full process, integrated with existing Identity and Access Management modules for internal security. Connecting this Balance System to the entity provider of the access to the Mexican digital banking system through a custom API and web services for incoming updates and a private network. All this implementation considers certain flexibility on the disbursement modalities and was developed through Object Oriented Programing Principles, managing models and controllers, to ensure the best programming practices and a correct integration with the current Business rules, processes, and models.
Keywords: Balance system, Buy Now Pay Later, disbursement, automation, compatibility
