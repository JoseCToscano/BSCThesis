%%%%%%%%%%%%%%%%%%%%%%%%%%%%
% PREAMBLE
%%%%%%%%%%%%%%%%%%%%%%%%%%%%

\chapter*{Preamble}

This paper reports general context, background, specifications and details on how the implementation of a balance system was designed and developed as the proposed automation for the main issue on discussion.


\subsection*{Preface}

As a mechatronics and automation engineering student the simplicity in today's technological solutions never ceases to amaze me. This simplicity lies within the understanding of what is really going on in every big or small engineering development. I like to use curiosity as my own drive to pursue that which is yet for me unknown and the desire of its understanding. I have always had the idea that we can not program something that we do not first completly understand, making the coding and the understanding both the means and the objective. This same applies for automation, where the intention will always be to replicate some process and its efficiency at larger scales, but only after this process has already become efficient and replicable itself. 

\subsection*{Acknowledgements}

Working in a startup has made this whole journey very challenging but enjoyable. It makes me really get a sense of responsibility through the tangibility of the results that are expected. This whole process could not have been possible without the support and effort of every single member not only the technological team, but also from the people bringing Atrato's operations to life day by day. I would like to thank Atrato for the opportunity of letting me keep pushing twoards the same goal, even from a different timezone. Being able to pursue a double degree abroad while at the same time contributing to offer great financial opportunities in my home country through my daily effort gives me a great sense of gratification. \\

Furthermore I would like to personally thank the founding team Juan, Roger and Alex for giving me the opportunity to learn and grow with them, it has been an increadible journey.

\subsection*{Notations and conventions}

Throught this paper, most techincal details are explained through tools such as flowcharts, tables, diagrams and UML. Code snippets are only included when specific insights need to be detailed. When making reference to objects and entities Pascal Case is used to note the difference; \textit{italics} are used to specify variables, or specific attributes or characteristics of these entities.

